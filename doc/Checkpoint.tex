\documentclass{article}
\usepackage[english]{babel}

%%%%%%%%%% Start TeXmacs macros
\newenvironment{itemizeminus}{\begin{itemize} \renewcommand{\labelitemi}{$-$}\renewcommand{\labelitemii}{$-$}\renewcommand{\labelitemiii}{$-$}\renewcommand{\labelitemiv}{$-$}}{\end{itemize}}
%%%%%%%%%% End TeXmacs macros

\begin{document}

\title{ARM Checkpoint }
\author{}
\maketitle

\section{Group Organisation}

We split the workload so that three of us would write the emulator and one
would write the assembler because we assumed that the emulator part would take
longer. The initial split was as follows:

Chris:
\begin{itemizeminus}
  \item Main pipeline
  
  \item Multiply
\end{itemizeminus}
\ \ \ \ Jixi:
\begin{itemizeminus}
  \item Single data transfer
  
  \item Shifted register for data processing
\end{itemizeminus}
\ \ \ \ Ryan:
\begin{itemizeminus}
  \item Rest of data processing
  
  \item Branch
\end{itemizeminus}
\ \ \ \ Shuhao:
\begin{itemizeminus}
  \item Assembler (part II)
\end{itemizeminus}
We also introduced a "to do" list so that after completing our initial splits,
we could pick out tasks from this list that needed to be done and implement
them.

For the emulator, the implementation strategy was very successful as we had
clear goals in what we needed to implement. As the group that coded the
emulator part was bigger, naturally, we finished part I earlier than part II.

The transition to part II was not too difficult as the team member
implementing the assembler part had written a grand design for the other
members to follow. This code however had few comments and so time was wasted
trying to decipher the code that was written. In hindsight, having comments in
the part II code earlier on would have helped the emulator group understand
the code faster and would have increased productivity.

We believe that using three members to implement part I was a good strategy as
every member of the sub-team had a task to do. Had we used all four members,
the workload could not have been distributed evenly amongst the members and
not every member could contribute to the work effort.

\section{Implementation Strategies}

In our design, we used global variable pointers to represent the main memory
and register file. We decided to use global variables (even though it is
usually bad practice as it is error prone) because we needed to access these
locations in many functions. Therefore, global pointers were used so that we
did not have to carry the pointers as function arguments which cluttered the
code. This made the code more readable and allowed for easy access to the main
memory and the register file. Clearly there is a trade-off between using
global variables and passing the pointers as arguments to functions and we
decided that there were more positives than negatives using global variables.

After consulting with the supervisors and course leader, we were advised to
change as many magic numbers into macros as possible through the \#define
declaration. Although we followed the advice given, we believe that this made
the code less readable and actually cluttered the code despite changing the
magic numbers into descriptive names. Avoiding magic numbers was simply
unavoidable in this exercise and so we have added as many comments as possible
to guide the readability of the code.

At the beginning of the project, the importance of code style was stressed.
Another thing that our team strived to do was to name functions and variables
appropriately. This, along with the rare use of type casting allowed for good
clarity of code. With every compilation of code, we used the flags -Wall,
-Werror and -pedantic. This ensured that our code compiled with no warnings as
the code would not compile (warnings would be converted to errors). We also
littered the code with useful comments that hopefully help the reader to
understand the code. All variables were also initialised to some value before
being used to avoid undefined behaviour. Curly brackets were used for every
control statement - even for single-command control statements.

We moved all function declarations into the emulate.h file and moved all
function definitions into the emulate.c file. The emulate.c file is ordered
with respect to the function declarations of emulate.h. Every effort was made
to ensure that the code style was consistent and that we had proper
indentation and consistent spaces (for example between brackets).

At first, we used assertions to handle errors. After a while however, we
realised that using if statements and using perror to print out an error was
more effective and so we adopted this use of error handling. The perror
statements are used every time a case was entered where it should not have
been. The -std=c99 was also used in compilation to ensure that we used ISO C99
and not ANSI C89. On top of this, if a function took no arguments, we used
void as a parameter of the function.

For the single data transfer instruction, we hit a stumbling block when
planning the implementation of the function. The function required the
checking of many conditions in the instruction, and a lot of thought went
behind reducing the number of if statements in the function. This problem was
overcome by using many helper functions to distinguish between the complicated
if statements.

Without realising we had made a mistake, we implemented the main memory to be
word (32-bits) addressable as opposed to byte (8-bits) addressable. When
testing the single data transfer instruction, we came to a realisation that
the function required an unaligned word access to the main memory which was
not possible with our implementation. After careful inspection of the
specification, we had a light-bulb moment that the main memory should have
been byte addressable and we were able to change our code to meet the
specification without having to change too much code.

We realised we could reuse certain utility functions from part I to part II -
namely createMask, getBits and switchEndian and this reduced code duplication
between the two parts.

We have just completed assembler (part II) and will be moving on to part III.
As a team, the biggest problem we will face in later tasks is thinking about
how we will split the workload amongst the team and how we will approach our
extension task once we come up with an extension idea. To try and resolve
these problems we will need to spend more time planning our individual
participations and comment our code effectively.

\end{document}
